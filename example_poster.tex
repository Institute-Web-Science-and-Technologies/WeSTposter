% !Mode:: "TeX:UTF-8"

% Enable warnings about problematic code
\RequirePackage[l2tabu, orthodox]{nag}

\documentclass{WeSTposter}

% ==============================================================================
% Metadata

\title{An Example of a WeST-Poster}
\author{Lukas Schmelzeisen}
\authormail{lukas@uni-koblenz.de}
\institute{Institute for Web Science and Technologies}


% ==============================================================================
% Packages

% Language typesetting.
\usepackage{polyglossia}
\setdefaultlanguage[
  variant = american, % Use American instead of Britsh English.
]{english}

% Unicode support for math. Load after other math/font packages.
\usepackage{unicode-math}

% Patches a few math packages to work correctly with LuaLaTeX.
\usepackage{lualatex-math}

% Various useful tools for mathematical typesetting.
\usepackage{mathtools}

% Quotations.
\usepackage[
  strict, % Turn warnings into errors.
]{csquotes}

% Bibliography.
\usepackage[
  backend = biber, % use biber as backend
  style = authoryear-comp,
%   style = alphabetic,
  maxbibnames = 10, % max number of names in bibliography
  maxcitenames = 2, % max number of names in text cite
  uniquelist = minyear, % only add more authors if year is not unique
  firstinits = true, % abbreviate first name of authors
  doi = false, % do not show doi
  isbn = false, % do not show isbn
  url = false, % do not show url
  eprint = false, % do not show eprint
  urldate = long, % format of urldate field
]{biblatex}
\addbibresource{example_bibliography.bib}
% Last names before first names: http://tex.stackexchange.com/q/113573
\DeclareNameAlias{sortname}{last-first}
\DeclareNameAlias{default}{last-first}
% Semicolons to separate authors: http://tex.stackexchange.com/q/197435
%\renewcommand{\multinamedelim}{\addsemicolon\space}
% Author lastnames in small caps, but only in bibliography
\AtBeginBibliography{
  \renewcommand\mkbibnamelast[1]{\textsc{#1}}
}
% Decrease bibliography font size
\def\bibfont{\small}

% ==============================================================================
% Editorial

\pdfcompresslevel=9
\pdfobjcompresslevel=2

\begin{document}
\frenchspacing

\maketitle

\begin{columns}
  \column{0.5} % ---------------------------------------------------------------
  \block{An Example of a Block}{
    This is an example of a scientific power using the WeST in the corporate
    identity design.
    Only blindtext follows in this column.

    Lorem ipsum dolor sit amet, consectetur adipiscing elit.
    Aliquam volutpat, leo ac malesuada malesuada, nulla sapien auctor neque, eu
    euismod libero metus ac justo.
    Praesent ipsum sem, sollicitudin vel quam in, scelerisque mollis velit.
    Sed eu dui vitae nisi auctor semper.
    Mauris sed eleifend neque, in consequat massa.
    Donec suscipit vestibulum purus dictum suscipit.
    Praesent porta venenatis risus, non congue ante aliquet non.

    Ut efficitur, nibh non posuere tristique, lectus tortor accumsan arcu, eget
    gravida mi libero at arcu.
    Nulla nec enim in ipsum aliquam sollicitudin.
    Nam feugiat consectetur tortor nec cursus.
    Aenean ut justo rhoncus, imperdiet velit nec, elementum ex.
    Praesent vestibulum elit eu consectetur mattis.
    Maecenas vitae finibus magna.
    Sed et rhoncus lorem.

    Suspendisse blandit nisl nec odio mollis commodo.
    Vestibulum vitae sapien ligula.
    Nam tincidunt massa vitae sem ultrices, ac placerat ante sodales.
  }
  \block{Another Block}{
    Ut interdum metus eu fringilla hendrerit.
    Nullam ut ultricies elit.
    Duis mauris maurs, faucibus quis posuere at, dignissim a diam.

    \begin{itemize}
      \item\textbf{Nam nec porta orci.}

        Morbi sed velit urna.
        Aenean venenatis tincidunt nisl id suscipit.
        Nunc quis dictum eros.

      \item\textbf{Suspendisse dolor turpis, molestie ac turpis quis, consequat
        laoreet nisl.}

        In nisl nibh, venenatis id risus ut, convallis aliquam lectus.
        Vivamus convallis lacinia sagittis.
        Sed quis molestie nisl.
        Morbi fermentum elementum ligula.

      \item\textbf{Vestibulum gravida ante magna.}

        Mauris sit amet diam eu neque iaculis semper sed nec lorem.
        Donec tempor nisl vitae feugiat eleifend.
        Duis commodo sem vitae vehicula tristique.
        Ut et ullamcorper est.
    \end{itemize}
  }

  \column{0.5} % ---------------------------------------------------------------
  \block{The Third Block}{
    This class also supports citations \parencite{DBLP:conf/leet/CheckowaySR10}.
    For example read \textcite{DBLP:books/sp/Gratzer16}, it may be interesting.
    More blindtext following.

    Etiam aliquam viverra tortor ut tincidunt.
    Praesent eget tincidunt tellus, in fermentum nibh.
    Maecenas nulla turpis, convallis sit amet velit non, tristique malesuada
    lacus.

    \innerblock{An Inner Block}{
      Phasellus ante nisi, molestie at metus et, lobortis iaculis massa.
      In scelerisque scelerisque odio, ut pellentesque tellus vulputate quis.
      Phasellus vel pulvinar nulla.

      An example of an inline figure:
      \begin{center}
        \begin{tikzpicture}
          \node[inner sep = 0] (icon_documents)
            {\includegraphics[height = 150pt]{icon_documents}};
          \node[inner sep = 0, right = 250pt of icon_documents] (icon_rdf)
            {\includegraphics[height = 150pt]{icon_rdf}};
          \draw[-{Triangle[length = 50pt, width = 120pt]}, line width = 50pt,
              draw = colorOne, shorten >= 50pt, shorten <= 50pt]
            (icon_documents.east) -- (icon_rdf.west);
        \end{tikzpicture}
      \end{center}

      Cras tincidunt tincidunt massa, non scelerisque risus.
      Vestibulum vel fermentum nisi.
      In dignissim euismod ex, accumsan cursus eros placerat placerat.

      \begin{tikzfigure}[This is how you do a separated figure with numbering.]
        \centering
        \begin{tikzpicture}[draw = black,
          gender/.style = {-{Latex[width = 30pt]},
            draw = colorOne, line width = 7pt},
          plural/.style = {-{Latex[width = 30pt]},
            draw = colorThree, line width = 7pt}]
          \begin{scope}
            \draw[fill = white] (0, 0) rectangle (500pt, 400pt);

            \node (man) at (60pt, 250pt) {man};
            \node (woman) at (260pt, 370pt) {woman};
            \draw[gender] (man) -- (woman);

            \node (uncle) at (240pt, 200pt) {uncle};
            \node (aunt) at (440pt, 320pt) {aunt};
            \draw[gender] (uncle) -- (aunt);

            \node (king) at (170pt, 30pt) {king};
            \node (queen) at (370pt, 150pt) {queen};
            \draw[gender] (king) -- (queen);
          \end{scope}
          \begin{scope}[shift = {(525pt, 0)}]
            \draw[fill = white] (0, 0) rectangle (500pt, 400pt);

            \node (king) at (170pt, 30pt) {king};
            \node (queen) at (370pt, 150pt) {queen};
            \draw[gender] (king) -- (queen);

            \node (kings) at (70pt, 230pt) {kings};
            \node (queens) at (270pt, 350pt) {queens};
            \draw[plural] (king) -- (kings);
            \draw[plural] (queen) -- (queens);
          \end{scope}
        \end{tikzpicture}
      \end{tikzfigure}
    }
  }
  \block{References}{
    \sloppy
    \printbibliography[heading = none]
  }

  % === Some left over blindtext in case we need more:
  % Nunc nunc est, semper sit amet sodales ac, hendrerit id lacus.
  % Praesent et lacus finibus, tempor odio vel, suscipit augue.
  % Morbi commodo risus pellentesque venenatis interdum.
  % Sed faucibus vel quam sed posuere.
  % Vestibulum non tempus nibh.
  % Suspendisse pulvinar ex sed lobortis consectetur.
  % Nam luctus eleifend urna et rutrum.
  % Curabitur et bibendum dui.
\end{columns}

\end{document}
